\documentclass[12pt]{article}

\usepackage{amsmath}
\usepackage{array}
\usepackage{amssymb}
\usepackage{geometry}[margin=0.5in]
\usepackage{enumitem}
\usepackage{color}
\usepackage{amsthm}
\usepackage{mathtools}

\theoremstyle{definition}
\newtheorem{definition}{Definition}[section]


\newcommand{\keyword}[1]{\textbf{\textcolor{red}{#1}}}
\newcommand{\subkeyword}[1]{\textbf{\textcolor{blue}{#1}}}
\newcommand{\mat}[1]{\textit{\textbf{#1}}}
\newcommand{\inverse}[1]{\textit{\textbf{#1}}^{-1}}
\newcommand{\transpose}[1]{\textit{\textbf{#1}}^{T}}
\newcommand{\rspace}[2]{\mathbb{R}^{#1 \times #2}}

\title{Linear Algebra}
\date{2020-11-21}
\author{Yuji Takai}

\begin{document}
    \pagenumbering{gobble}
    \maketitle
    \newpage
    \section{Introduction}
        \begin{definition}[\keyword{Vector}]
            Object that can be added together and 
                multiplied by scalars to produce another object of the same kind
        \end{definition}
        \begin{itemize}
            \item Examples: geometric vectors, polynomials, audio 
                signals, elements of $\mathbb{R}^n$
        \end{itemize}
    \section{System of linear equations}
        \begin{align*}
            a_{11}x_1 + \cdots &+ a_{1n} = b_1\\
            \vdots&\\
            a_{m1}x_1 + \cdots &+ a_{mn} = b_m
        \end{align*}
        \begin{itemize}
            \item Every $n$-tuple $(x_1, \cdots, x_n) \in \mathbb{R}^n$ that 
                satisfies the system of linear equations is a solution of the
                linear equation system
            \item Either \textit{no, exactly one, or infinitely many} solutions can be 
                obtained for a real-valued system of linear equations
            \item System of lilnear equation into matrix multiplication form
            \begin{equation*}
                \left[
                    \begin{matrix}
                        a_{11} & \cdots & a_{1n}\\
                        \vdots & & \vdots\\
                        a_{m1} & \cdots & a_{mn}
                    \end{matrix}
                \right]
                \left[
                    \begin{matrix}
                        x_1\\
                        \vdots\\
                        x_n
                    \end{matrix}
                \right]
                =
                \left[
                    \begin{matrix}
                        b_1\\
                        \vdots\\
                        b_m
                    \end{matrix}
                \right]
            \end{equation*}
        \end{itemize}
    \newpage
    \section{Matrices}
        \subsection{Definitions}
            \begin{definition}[\keyword{Matrix}]
                $m\times n$ tuple of elements $a_{ij}$ ($m, n \in \mathbb{R}$, 
                $i = 1, \cdots ,m$ \& $j = 1, \cdots, n$), which is ordered 
                according to a rectangular scheme consisting of $m$ rows and 
                $n$ columns
                \begin{equation*}
                    \mat{A} = \left[
                        \begin{matrix}
                            a_{11} & a_{12} & \cdots & a_{1n}\\
                            a_{21} & a_{22} & \cdots & a_{2n}\\
                            \vdots & \vdots &        & \vdots\\
                            a_{m1} & a_{m2} & \cdots & a_{mn}
                        \end{matrix}
                    \right], \quad a_{ij} \in \mathbb{R}
                \end{equation*}
                \begin{itemize}
                    \item Represents systems of linear equations and linear 
                        functions (linear mappings)
                \end{itemize}
            \end{definition}
            \begin{definition}[\keyword{Identity Matrix}] 
                Matrix $\mat{I}_n \in \mathbb{R}^{n\times n}$ that contains $1$ on the diagonal 
                and $0$ everywhere else
                \begin{equation*}
                    \mat{I}_n \coloneqq \left[
                        \begin{matrix}
                            1      & 0      & \cdots & 0      & \cdots & 0      \\
                            0      & 1      & \cdots & 0      & \cdots & 0      \\
                            \vdots & \vdots & \ddots & \vdots & \ddots & \vdots \\
                            0      & 0      & \cdots & 1      & \cdots & 0      \\
                            \vdots & \vdots & \ddots & \vdots & \ddots & \vdots \\
                            0      & 0      & \cdots & 0      & \cdots & 1      
                        \end{matrix}
                    \right] \in \mathbb{R}^{n\times n}
                \end{equation*}
            \end{definition}
            \begin{definition}[\keyword{Inverse}]
                For $\mat{A}\in\rspace{n}{n}$, $\mat{B}\in\rspace{n}{n}$ 
                that has the property $\mat{A}\mat{B}=\mat{I}_n=\mat{B}\mat{A}$ 
                is the \keyword{inverse} of $\mat{A}$, denoted by $\inverse{A}$
                \begin{itemize}
                    \item If $\mat{A}$ is \keyword{regular}/\keyword{invertible}/
                        \keyword{nonsingular}, $\inverse{A}$ exists
                    \item Otherwise $\mat{A}$ is \keyword{singular}/\keyword{noninvertible}
                    \item When $\inverse{A}$ exists, it is unique
                    \item Can use the \keyword{determinant} to check whether a
                        matrix is invertible
                    \item If $\mat{A}$ is invertible, then so is $\transpose{A}$, 
                        and $\mat{A}^{-T}\coloneqq (\inverse{A})^{-T} = (\transpose{A})^{-1}$
                    \item To compute the inverse, apply the \keyword{Gaussian elimination}
                        on the augmented matrix $[\mat{A}|\mat{I}_n]$ to obtain 
                        the reduced-row echelon form $[\mat{I}_n|\inverse{A}]$
                \end{itemize}
            \end{definition}
            \begin{definition}[\keyword{Transpose}]
                For $\mat{A}\in\rspace{m}{n}$, $\mat{B}\in\rspace{n}{m}$ with
                $b_{ij} = a_{ji}$ is the \keyword{transpose} of $\mat{A}$, 
                denoted by $\transpose{A}$ 
            \end{definition}
            \begin{definition}[\keyword{Symmetric Matrix}]
                $\mat{A}\in\rspace{n}{n}$ such that $\mat{A}=\transpose{A}$
                \begin{itemize}
                    \item Sum of symmetric matrices is always symmetric
                    \item Product is always defined, but generally not symmetric
                \end{itemize}
            \end{definition}
        \subsection{Operations on Matrix}
            \begin{definition}[\subkeyword{Matrix Addition}]
                Element-wise sum for $\mat{A}, \mat{B} \in \rspace{m}{n}$
                \begin{equation*}
                    \mat{A} + \mat{B} \coloneqq \left[
                        \begin{matrix}
                            a_{11} + b_{11} & \cdots & a_{1n} + b_{1n}\\
                            \vdots          &        & \vdots         \\
                            a_{m1} + b_{m1} & \cdots & a_{mn} + b_{mn}\\
                        \end{matrix}
                    \right] \in \rspace{m}{n}
                \end{equation*}
            \end{definition}
            \begin{definition}[\subkeyword{Matrix Multiplication}]
                Matrix multiplication of $\mat{A} \in \rspace{m}{n}$, 
                $\mat{B} \in \mathbb{R}^{n\times k}$ results in $\mat{C}\in 
                \mathbb{R}^{\times k}$ with each element $c_{ij}$ being 
                \subkeyword{dot product} of the $i$th row of $\mat{A}$ and $j$th 
                column of $\mat{B}$
                \begin{equation*}
                    c_{ij} = \sum_{l = 1}^{n}a_{il}b_{lj} \quad 
                    (i=1,\cdots,m, \quad j=1,\cdots,k)
                \end{equation*}
                \begin{itemize}
                    \item Matrices can only be multiplied if their 
                        “neighboring” dimensions match. 
                    \item Ex: an $n\times k$-matrix $\mat{A}$ can be 
                        multiplied with a $k\times m$-matrix $\mat{B}$, 
                        but only from the left side i.e. 
                        $\mat{A}\mat{B} = \mat{C}$, and $\mat{B}\mat{A}$ is
                        undefined
                    \item NOT defined as an element-wise operation on matrix
                        elements i.e. $c_{ij} \neq a_{ij}b_{ij}$
                \end{itemize}
            \end{definition}
            \begin{definition}[\subkeyword{Hadamard product}]
                Element-wise multiplication of equally-sized multi-dimensional arrays
            \end{definition}
            \begin{definition}[\subkeyword{Scalar multiplication}]
                If $\mat{A}\in\rspace{m}{n}$ and $\lambda \in \mathbb{R}$, then 
                $\lambda \mat{A}=\mat{K}$ where $K_{ij} = \lambda a_{ij}$
            \end{definition}
        \subsection{Properties of Matrix}
            \begin{itemize}
                \item \subkeyword{Associativity}:
                    $\forall \mat{A}\in\rspace{m}{n}$, 
                    $\mat{B}\in\rspace{n}{p}$, 
                    $\mat{C}\in\rspace{p}{q}$: 
                    $(\mat{A}\mat{B})\mat{C} = \mat{A}(\mat{B}\mat{C})$
                \item \subkeyword{Distributivity}:
                    $\forall \mat{A}, \mat{B}\in\rspace{m}{n}$, 
                    $\mat{C}, \mat{D}\in\mathbb{R}^{n\times p}$:
                    \begin{itemize}
                        \item $(\mat{A}+\mat{B})\mat{C}=\mat{A}\mat{C}+\mat{B}\mat{C}$
                        \item $\mat{A}(\mat{C}+\mat{D})=\mat{A}\mat{C}+\mat{A}\mat{D}$
                    \end{itemize} 
                \item Multiplication with $\mat{I}$:
                    $\forall \mat{A}\in\rspace{m}{n}$:
                    $\mat{I}_m\mat{A}=\mat{A}\mat{I}_n=\mat{A}$
                \item Inverse 
                    \begin{itemize}
                        \item $\mat{A}\inverse{A}=\mat{I}=\inverse{A}\mat{A}$
                        \item $(\mat{A}\mat{B})^{-1} = \inverse{B}\inverse{A}$
                        \item $(\mat{A}+\mat{B})^{-1} \neq \inverse{A} + \inverse{B}$
                    \end{itemize}
                \item Transpose 
                    \begin{itemize}
                        \item $\left(\transpose{A}\right)^T=\mat{A}$
                        \item $(\mat{A}+\mat{B})^T=\transpose{A}+\transpose{B}$
                        \item $(\mat{A}\mat{B})^T=\transpose{B}\transpose{A}$
                    \end{itemize}
                \item Multiplication with scalar 
                    \begin{itemize}
                        \item \subkeyword{Associativity}
                            \begin{itemize}
                                \item $(\lambda \psi)\mat{C}=\lambda(\psi\mat{C}), \quad \mat{C}\in\rspace{m}{n}$
                                \item $\lambda (\mat{B}\mat{C})=(\lambda\mat{B})\mat{C}=\mat{B}(\lambda\mat{C})=(\mat{B}\mat{C})\lambda, \quad \mat{B}\in\rspace{m}{n},\mat{C}\in\rspace{n}{k}$
                                \item $(\lambda \mat{C})^{T}=\transpose{C}\lambda^T=\transpose{C}\lambda=\lambda\transpose{C}$ (since $\forall\lambda\in\mathbb{R}: \lambda = \lambda^T$)
                            \end{itemize}
                        \item \subkeyword{Distributivity}
                            \begin{itemize}
                                \item $(\lambda+\psi)\mat{C}=\lambda\mat{C}+\psi\mat{C}, \quad\mat{C}\in\rspace{m}{n}$
                                \item $\lambda(\mat{B}+\mat{C})=\lambda\mat{B}+\lambda\mat{C}, \quad\mat{B},\mat{C}\in\rspace{m}{n}$
                            \end{itemize}
                    \end{itemize}
            \end{itemize}
        \newpage
        \subsection{Row reduction}
            \begin{definition}[\subkeyword{Elementary transformation}]
                Transformation of the system of equation into a simpler form while
                keeping the solution set the same
                \begin{itemize}
                    \item Exchange of 2 equations (rows in the matrix representing
                        the system of equations)
                    \item Multiplcation of an equation (row) with a constant 
                        $\lambda\in\mathbb{R}\symbol{92}\{0\}$
                    \item Addition of 2 equations (rows)
                \end{itemize}
            \end{definition}
            \begin{definition}[\subkeyword{Pivot}]
                Leading coefficient of a row and is always strictly to the right 
                of the pivot of the row above it
                \begin{itemize}
                    \item any equation system in row-echelon form always has a 
                        "staircase" structure
                \end{itemize}
            \end{definition}
            \begin{definition}[\subkeyword{Row-echelon form}]
                A matrix is in row-echelon form if:
                \begin{itemize}
                    \item All rows that contain only zeros are at the bottom of
                        the matrix i.e. all rows that contain at least one 
                        nonzero element are on top of rows that contain only zeros
                    \item Looking at nonzero rows only, the first nonzero number
                        from the left (\subkeyword{pivot}/ leading coefficient) 
                        is always strictly to the right of the pivot of the row
                        above it
                \end{itemize}
            \end{definition}
            \begin{definition}[\subkeyword{Basic variable}]
                Variables corresponding to the pivots in the 
                \subkeyword{row-echelon} form
            \end{definition}
            \begin{definition}[\subkeyword{Free variable}]
                Variables corresponding to the non-pivots in the 
                \subkeyword{row-echelon} form
            \end{definition}
            \begin{definition}[\subkeyword{Reduced row-echelon form}]
                A matrix is in reduced row-echelon form if:
                \begin{itemize}
                    \item it is in \subkeyword{row-echelon} form
                    \item every pivot is $1$
                    \item the pivot is the only nonzero entry in its column
                \end{itemize} 
            \end{definition}
            \begin{definition}[\subkeyword{Gaussian elimination}]
                An algorithm that performs elementary transformations to bring a
                system of linear equations into reduced row-echelon form. Useful
                for:
                \begin{itemize}
                    \item Computing determinants
                    \item Checking whether a set of vectors is \keyword{linearly independent}
                    \item Computing the inverse of a matrix 
                    \item Computing the rank of a matrix 
                    \item Determining a basis of a vector space
                \end{itemize}
                However, impractical for systems with millions of variables 
                ($O(n^3)$ algorithm)
            \end{definition}
        \subsection{Algorithms for solving a system of linear equations}
            \begin{equation*}
                \mat{A}\mat{x}=\mat{b}
            \end{equation*}
            \begin{itemize}
                \item If there is \textit{no solution}, need to resort to approximate 
                    solutions like \keyword{linear regression}
                \item If $\mat{A}$ is a \subkeyword{square matrix} and 
                    \keyword{invertible}, can determine $\inverse{A}$ and solve 
                    the equation via $\mat{x}=\inverse{A}\mat{b}$
                \item If $\mat{A}$ has linearly independent columns:
                    \begin{equation*}
                        \mat{A}\mat{x}=\mat{b}\Leftrightarrow
                        \transpose{A}\mat{A}\mat{x}=\transpose{A}\mat{b}\Leftrightarrow
                        \mat{x}=(\transpose{A}\mat{A})^{-1}\transpose{A}\mat{b}
                    \end{equation*}
                    \begin{itemize}
                        \item \subkeyword{Moore-Penrose pseudo-inverse} $(\transpose{A}\mat{A})^{-1}\transpose{A}$
                        \item Solution corresponds to the minimum norm least-squares solution
                        \item Requires many computations for the matrix-matrix 
                            product and computating inverse of $\transpose{A}\mat{A}$
                        \item For reasons of numerical precision, not recommended 
                            to compute the inverse or pseudo-inverse 
                    \end{itemize}
                \item Practical methods that indirectly solve systems of many linear equations
                    \begin{itemize}
                        \item Stationary iterative methods: Richardson  method, Jacobi method, Gauß-Seidel method, successive over-relaxation method 
                        \item Krylov subspace methods: conjugate gradients, generalized minimal residual, biconjugate gradients
                        \item Key idea of iterative methods: set up an iteration of the form
                            \begin{equation*}
                                \mat{x}^{(k+1)}=\mat{C}\mat{x}^{(k)}+\mat{d}
                            \end{equation*}
                            for suitable $\mat{C}$ and $\mat{d}$ that reduces 
                            the residual error $\|\mat{x}^{(k+1)}-\mat{x}_*\|$ 
                            in every iteration and converges to $\mat{x}_*$
                    \end{itemize}
            \end{itemize}
        
    \section{Vector spaces}
        \begin{definition}[\keyword{Group}]
            For a set $\mathcal{G}$ and an operation 
            $\otimes: \mathcal{G}\times\mathcal{G}\rightarrow\mathcal{G}$,
            $G\coloneqq(\mathcal{G}, \otimes)$ is called a \keyword{group} if 
            the following hold:
            \begin{enumerate}
                \item \subkeyword{Closure} of $\mathcal{G}$ under $\otimes$:
                    $\forall x, y\in\mathcal{G}: x\otimes y\in\mathcal{G}$
                \item \subkeyword{Associativity}: $\forall x,y,z\in\mathcal{G}: (x\otimes y)\otimes z=x\otimes(y\otimes z)$
                \item \subkeyword{Neutral element}: $\exists e\in\mathcal{G}, \forall x\in\mathcal{G}: x\otimes e=x \text{ and } e\otimes x=x$
                \item \subkeyword{Inverse element}: $\forall x\in\mathcal{G}, \exists y\in\mathcal{G}: x\otimes y=e \text{ and } y\otimes x=e$ ($x^{-1}$: inverse element of $x$)
            \end{enumerate}
        \end{definition}
        \begin{definition}[\keyword{Abelian group}]
            Group $G=(\mathcal{G},\otimes)$ with \subkeyword{commutative} property: 
            $\forall x,y\in\mathcal{G}$: $x\otimes y=y\otimes x$ 
        \end{definition}
        \begin{definition}[\keyword{General Linear Group}]
            The group with the set of regular/ invertible matrices 
            $\mat{A}\in\rspace{n}{n}$ and matrix multiplication as operation, 
            denoted by $GL(n, \mathbb{R})$
            \begin{itemize}
                \item NOT Abelian $\leftarrow$ since matrix multiplication is 
                    NOT commutative
            \end{itemize}
        \end{definition}
        \begin{definition}[\keyword{Vector space}]
            $V=(\mathcal{V},+,\cdot)$ that is a set $\mathcal{V}$ with 2 operations
            \begin{itemize}
                \item $+$: $\mathcal{V}\times\mathcal{V}\rightarrow\mathcal{V}$ 
                    (\subkeyword{vector addition}/\subkeyword{inner operation})
                \item $\cdot$: $\mathbb{R}\times\mathcal{V}\rightarrow\mathcal{V}$ 
                    (\subkeyword{multiplication by scalars}/\subkeyword{outer operation})
            \end{itemize}
            where:
            \begin{enumerate}
                \item $(\mathcal{V},+)$ is an Abelian group
                \item \subkeyword{Distributivity}:
                    \begin{itemize}
                        \item $\forall\lambda\in\mathbb{R},\mat{x},\mat{y}\in\mathcal{V}$: 
                            $\lambda\cdot(\mat{x}+\mat{y})=\lambda\cdot\mat{x}+\lambda\cdot\mat{y}$
                        \item $\forall\lambda,\psi\in\mathbb{R},\mat{x}\in\mathcal{V}$: 
                            $(\lambda+\psi)\cdot\mat{x}=\lambda\cdot\mat{x}+\psi\cdot\mat{x}$
                    \end{itemize}
                \item \subkeyword{Associativity}: 
                    $\forall\lambda,\psi\in\mathbb{R},\mat{x}\in\mathcal{V}$: 
                    $\lambda\cdot(\psi\cdot\mat{x})=(\lambda\psi)\cdot\mat{x}$
                \item \subkeyword{Neutral element} w.r.t the outer operation:
                    $\forall\mat{x}\in\mathcal{V}$: $1\cdot\mat{x}=\mat{x}$
            \end{enumerate}
            Definitions/Properties:
            \begin{enumerate}
                \item \keyword{Vectors}: the elements $\mat{x}\in\mat{V}$
                \item \keyword{Scalars}: the elements $\lambda\in\mathbb{R}$
                \item Neutral element of $(\mathcal{V}, +)$ = zero vector $\mat{0}$
            \end{enumerate}
        \end{definition}
        \newpage
        \begin{definition}[\keyword{Vector subspace}]
            Let $V=(\mathcal{V},+,\cdot)$ and $\mathcal{U}\subseteq\mathcal{V}$, 
            $\mathcal{U}\neq\emptyset$. Then $U=(\mathcal{U},+,\cdot)$ is 
            \keyword{vector subspace}/\keyword{linear subspace} of $V$ if $U$ is
            a vector space with the vector space operations $+$ and $\cdot$ 
            restricted to $\mathcal{U}\times\mathcal{U}$ and 
            $\mathbb{R}\times\mathcal{U}$, denoted as $U\subseteq V$.
            To determine $U=(\mathcal{U}, +, \cdot)$ is a subspace of $V$:
            \begin{enumerate}
                \item $\mathcal{U}\neq\emptyset$, in particular: 
                    $\mat{0}\in\mathcal{U}$
                \item Closure of $U$:
                    \begin{itemize}
                        \item w.r.t the outer operation: $\forall\lambda\in\mathbb{R},\forall\mat{x}\in\mathcal{U}$: $\lambda\mat{x}\in\mathcal{U}$
                        \item w.r.t the inner operation: $\forall\mat{x},\mat{y}\in\mathcal{U}$: $\mat{x}+\mat{y}\in\mathcal{U}$
                    \end{itemize}
            \end{enumerate}
            NOTE: every subspace $U\subseteq(\mathbb{R}^n, +, \cdot)$ is the 
            solution space of a homogeneous system of linear equations 
            $\mat{A}\mat{x}=\mat{0}$ for $\mat{x}\in\mathbb{R}^n$
        \end{definition}
    \section{}
    \section{}
    \section{}
    \section{}

\end{document}