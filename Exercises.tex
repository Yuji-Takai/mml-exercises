\documentclass[12pt]{article}

\usepackage{amsmath}
\usepackage{array}
\usepackage{amssymb}
\usepackage{geometry}[margin=0.5in]
\usepackage{enumitem}

\title{Exercises}
\date{2020-11-15}
\author{Yuji Takai}

\begin{document}
    \pagenumbering{gobble}
    \maketitle
    \newpage
    \section{Chapter 1}
    
    \section{Chapter 2}
        \subsection{}
        \subsection{}
        \subsection{}
        \subsection{}
        \subsection{}
        \subsection{}
        \subsection{}
        \newpage
        \subsection{}
            \begin{enumerate}[label=(\alph*)]
                \item Row reduce the augmented matrix $[A | I]$ to determine whether there exists an inverse for $A$
                    \begin{align*}
                        [A | I] &= 
                        \left[\begin{array}{ccc|ccc}
                            2 & 3 & 4 & 1 & 0 & 0\\
                            3 & 4 & 5 & 0 & 1 & 0\\
                            4 & 5 & 6 & 0 & 0 & 1
                        \end{array}\right] \\
                        &\sim
                        \left[\begin{array}{ccc|ccc}
                            2 & 3 & 4 & 1 & 0 & 0\\
                            0 & 1 & 2 & 3 & -2 & 0\\
                            0 & 0 & 0 & 1 & -2 & 1
                        \end{array}\right] \\
                    \end{align*}
                    Since the third column of the augmented matrix is not a pivot column, $A^{-1}$ does not exist
                \item Row reduce the augmented matrix $[A | I]$ to determine whether there exists an inverse for $A$
                    \begin{align*}
                        [A | I] &= 
                        \left[\begin{array}{cccc|cccc}
                            1 & 0 & 1 & 0 & 1 & 0 & 0 & 0\\
                            0 & 1 & 1 & 0 & 0 & 1 & 0 & 0\\
                            1 & 1 & 0 & 1 & 0 & 0 & 1 & 0\\
                            1 & 1 & 1 & 0 & 0 & 0 & 0 & 1\\
                        \end{array}\right] \\
                        &\sim
                        \left[\begin{array}{cccc|cccc}
                            1 & 0 & 0 & 0 & 0 & -1 & 0 & 1\\
                            0 & 1 & 0 & 0 & -1 & 0 & 0 & 1\\
                            0 & 0 & 1 & 0 & 1 & 1 & 0 & -1\\
                            0 & 0 & 0 & 1 & 1 & 1 & 1 & -2\\
                        \end{array}\right] \\
                        \therefore A^{-1} &=  
                        \left[\begin{matrix}
                            0 & -1 & 0 & 1\\
                            -1 & 0 & 0 & 1\\
                            1 & 1 & 0 & -1\\
                            1 & 1 & 1 & -2\\
                        \end{matrix}\right]
                    \end{align*}
            \end{enumerate}
        \newpage
        \subsection{}
            \begin{enumerate}[label=(\alph*)]
                \item To prove $A$ is a subpace of $\mathbb{R}^3$, we need to prove that $A$ is not an empty set and $A$ is closed with respect to both inner and outer
                    \begin{enumerate}[label=\arabic*.]
                        \item Let $\vec{v}\in A$. When $\lambda = 0$, $\mu = 0$, $\vec{v} = (0, 0, 0)$. This means that $\textbf{0}\in A$ and thus $A \neq \emptyset$
                        \item Closure of $A$
                            \begin{itemize}
                                \item With respect to the outer operation: 
                                    $\forall c\in \mathbb{R}, \forall\vec{x}\in A: c\vec{x}\in A$
                                    \begin{itemize}
                                        \item $c\vec{x} = (c\lambda, c(\lambda + \mu^3), c(\lambda - \mu^3))\in A$
                                    \end{itemize} 
                                \item With respect to the inner operation: $\forall \vec{x}, \vec{y} \in A : \vec{x} + \vec{y} \in A$
                                    \begin{itemize}
                                        \item Let $\vec{x} = (\lambda_1, \lambda_1 + \mu_1^3, \lambda_1 - \mu_1^3)\in A$
                                        \item Let $\vec{y} = (\lambda_2, \lambda_2 + \mu_2^3, \lambda_2 - \mu_2^3)\in A$
                                        \item $\vec{x} + \vec{y} = (\lambda_1 + \lambda_2, \lambda_1 + \mu_1^3 + \lambda_2 + \mu_2^3, \lambda_1 - \mu_1^3 + \lambda_2 - \mu_2^3)$
                                        \item Let $\lambda_3 = \lambda_1 + \lambda_2$
                                        \item Then: $\vec{x} + \vec{y} = (\lambda_3, \lambda_3 + \mu_1^3 + \mu_2^3, \lambda_3 - \mu_1^3 - \mu_2^3)$
                                        \item Let $\mu_1^3 + \mu_2^3 = \mu_3^3$
                                        \item Then: $\vec{x} + \vec{y} = (\lambda_3, \lambda_3 + \mu_3^3, \lambda_3 - \mu_3^3) \in A$
                                    \end{itemize}
                            \end{itemize} 
                    \end{enumerate}
                \item To prove $B$ is a subpace of $\mathbb{R}^3$, we need to prove that $B$ is not an empty set and $B$ is closed with respect to both inner and outer
                    \begin{itemize}
                        \item While $B$ is not empty ($\vec{x} = (0, 0, 0) \in B$), $B$ is not closed with respect to outer operation
                        \item Let $\vec{x} = (1, -1, 0) \in B$ and $c = -1$. Then $c\vec{x} = (-1, 1, 0)$. Since there is no number in $\mathbb{R}$ such that its square is less than 0. (only complex numbers are possible). 
                        \item Therefore $c\vec{x} \notin B$
                    \end{itemize}
                \item To prove $C$ is a subpace of $\mathbb{R}^3$, we need to prove that $C$ is not an empty set and $C$ is closed with respect to both inner and outer
                    \begin{enumerate}[label=\arabic*.]
                        \item Let $\vec{v}\in C$. When $\xi_1 = 0$, $\xi_2 = 0$, $\xi_3 = 0$, $\vec{v} = (0, 0, 0) with \gamma = 0$. This means that $\textbf{0}\in C$ and thus $C \neq \emptyset$
                        \item Closure of $C$
                            \begin{itemize}
                                \item With respect to the outer operation: 
                                    $\forall c\in \mathbb{R}, \forall\vec{x}\in A: c\vec{x}\in A$
                                    \begin{itemize}
                                        \item $c\vec{x} = c(\xi_1, \xi_2, \xi_3) = (\xi'_1, \xi'_2, \xi'_3)$ with $\xi'_1 = c\xi_1, \xi'_2 = c\xi_2, \xi'_3 = c\xi_3, \xi'_1 + \xi'_2 + \xi'_3 = c\gamma = \gamma'$
                                        \item Therefore closed wrt outer operation
                                    \end{itemize} 
                                \item With respect to the inner operation: $\forall \vec{x}, \vec{y} \in C : \vec{x} + \vec{y} \in C$
                                    \begin{itemize}
                                        \item Since of addition of a number in $\mathbb{R}$ will always yield a number in $\mathbb{R}$, $\vec{x} + \vec{y} \in C$ if $\vec{x}, \vec{y} \in C$
                                    \end{itemize}
                            \end{itemize} 
                    \end{enumerate}
                \item To prove $D$ is a subpace of $\mathbb{R}^3$, we need to prove that $D$ is not an empty set and $D$ is closed with respect to both inner and outer
                    \begin{itemize}
                        \item While $D$ is not empty ($\vec{x} = (0, 0, 0) \in B$), $D$ is not closed with respect to outer operation
                        \item Let $\vec{x} = (0, 1, 0) \in D$ and $c = 0.5$. Then $c\vec{x} = (0, 0.5, 0)$.
                        \item Since $0.5 \notin \mathbb{R}$, $c\vec{x} \notin D$
                    \end{itemize}
            \end{enumerate}
        \newpage
        \subsection{}
            \begin{enumerate}[label=(\alph*)]
                \item Let $A = [x_1, x_2, x_3]$ and row reduce it to see if all columns are pivots
                    \begin{align*}
                        A &= \left[\begin{matrix}
                            2 & 1 & 3\\
                            -1 & 1 & -3\\
                            3 & -2 & 8
                        \end{matrix}\right]\\
                        &\sim \left[\begin{matrix}
                            1 & 0 & 2\\
                            0 & 1 & -1\\
                            0 & 0 & 0
                        \end{matrix}\right]
                    \end{align*}
                    \begin{itemize}
                        \item Since the third column does not have a pivot, the three vectors are not linearly independent 
                        \item From the row reduced matrix form, we can find that $x_3 = 2x_1 + (-1x_2)$
                    \end{itemize}
                \item Let $A = [x_1, x_2, x_3]$ and row reduce it to see if all columns are pivots
                    \begin{align*}
                        A &= \left[\begin{matrix}
                            1 & 1 & 1\\
                            2 & 1 & 0\\
                            1 & 0 & 0\\
                            0 & 1 & 1\\
                            0 & 1 & 1
                        \end{matrix}\right]\\
                        &\sim \left[\begin{matrix}
                            1 & 0 & 0\\
                            0 & 1 & 0\\
                            0 & 0 & 1\\
                            0 & 0 & 0\\
                            0 & 0 & 0
                        \end{matrix}\right]
                    \end{align*}
                    \begin{itemize}
                        \item Since all columns of $A$ have a pivot, the three vectors are linearly independent
                    \end{itemize}
            \end{enumerate}
        \newpage
        \subsection{}
            Let $A = [x_1, x_2, x_3]$. We must find $x$ such that $Ax = y$. We will row reduce the augmented matrix $[A | y]$.
            \begin{align*}
                [A | y] &= \left[\begin{array}{ccc|c}
                    1 & 1 & 2 & 1\\
                    1 & 2 & -1 & -2\\
                    1 & 3 & 1 & 5
                \end{array}\right]\\
                &\sim \left[\begin{array}{ccc|c}
                    1 & 0 & 0 & -6\\
                    0 & 1 & 0 & 3\\
                    0 & 0 & 1 & 2
                \end{array}\right]
            \end{align*}
            We can verify that $-6x_1 + 3x_2 + 2x_3 = y$
        \subsection{}
            When a vector $\vec{v} \in U_1 \cap U_2$, then there must be $\vec{x}$ and $\vec{y}$ such that $U_1\vec{x} = U_2\vec{y} = \vec{v}$. It is clear that $\vec{v}$ will be a linear combination of the basis of $U_1 \cap U_2$. 
            We can get that $U_1\vec{x} - U_2\vec{y} = \vec{0}$. Let $A = [U_1 -U_2]$. Let's solve for the homogenous equation $A\vec{x'} = \vec{0}$:
            \begin{align*}
                A &= \left[\begin{matrix}
                    1 & 2 & -1 & 1 & -2 & 3\\
                    1 & -1 & 1 & 2 & 2 & -6\\
                    -3 & 0 & -1 & -2 & 0 & 2\\
                    1 & -1 & 1 & -1 & 0 & 1\\
                \end{matrix}\right]\\
                &\sim \left[\begin{matrix}
                    9 & 0 & 3 & 0 & -4 & 8\\
                    0 & 9 & -6 & 0 & -10 & 20\\
                    0 & 0 & 0 & 3 & 2 & -7\\
                    0 & 0 & 0 & 0 & 0 & 0\\
                \end{matrix}\right]\\
            \end{align*}
            We can easily find that the nullspace of $A$ is 
            \begin{equation*}
                span\left(
                    \left[\begin{matrix}
                        -1\\
                        2\\
                        3\\
                        0\\
                        0\\
                        0
                    \end{matrix}\right], 
                    \left[\begin{matrix}
                        4\\
                        10\\
                        0\\
                        -6\\
                        9\\
                        0
                    \end{matrix}\right],
                    \left[\begin{matrix}
                        -8\\
                        -20\\
                        0\\
                        21\\
                        0\\
                        9
                    \end{matrix}\right]
                \right)
            \end{equation*}
            Now, the interesting part of these three vectors is that, $\vec{x}$ is a linear combination of the upper half of the 3 vectors in the basis and $\vec{y}$ is the linear combination of the lower half of the 3 vectors in the basis.
            To get the basis of $U_1 \cap U_2$, we simply need to compute the result of multiplying $U_1$ and 3 vectors that are the upper half of the 3 vectors in the basis of nullspace of A.
            \begin{align*}
                \left[\begin{matrix}
                    1 & 2 & -1\\
                    1 & -1 & 1\\
                    -3 & 0 & -1\\
                    1 & -1 & 1
                \end{matrix}\right]
                \left[\begin{matrix}
                    -1\\
                    2\\
                    3\\
                \end{matrix}\right] &=
                \left[\begin{matrix}
                    0\\
                    0\\
                    0\\
                    0
                \end{matrix}\right]\\
                \left[\begin{matrix}
                    1 & 2 & -1\\
                    1 & -1 & 1\\
                    -3 & 0 & -1\\
                    1 & -1 & 1
                \end{matrix}\right]
                \left[\begin{matrix}
                    4\\
                    10\\
                    0\\
                \end{matrix}\right] &=
                \left[\begin{matrix}
                    24\\
                    -6\\
                    -12\\
                    -6
                \end{matrix}\right]\\
                \left[\begin{matrix}
                    1 & 2 & -1\\
                    1 & -1 & 1\\
                    -3 & 0 & -1\\
                    1 & -1 & 1
                \end{matrix}\right]
                \left[\begin{matrix}
                    -8\\
                    -20\\
                    0\\
                \end{matrix}\right] &=
                \left[\begin{matrix}
                    -48\\
                    12\\
                    24\\
                    12
                \end{matrix}\right]\\
            \end{align*}
            This means that: 
            \begin{equation*}
                U_1 \cap U_2 = 
                span\left(
                    \left[\begin{matrix}
                        0\\
                        0\\
                        0\\
                        0
                    \end{matrix}\right], 
                    \left[\begin{matrix}
                        24\\
                        -6\\
                        -12\\
                        -6
                    \end{matrix}\right],
                    \left[\begin{matrix}
                        -48\\
                        12\\
                        24\\
                        12
                    \end{matrix}\right]
                \right)
            \end{equation*}
            Therefore the basis of $U_1 \cap U_2$ is:
            \begin{equation*}
                \left[\begin{matrix}
                    4\\
                    -1\\
                    -2\\
                    -1
                \end{matrix}\right]
            \end{equation*}
        \newpage
        \subsection{}
            Before approaching the subquestions, we should solve for $U_1$ and $U_2$
            \begin{align*}
                [A_1 | \vec{0}] &= \left[\begin{array}{c c c | c}
                    1 & 0 & 1 & 0\\
                    1 & -2 & -1 & 0\\
                    2 & 1 & 3 & 0\\
                    1 & 0 & 1 & 0
                    \end{array}\right]\\
                    &\sim \left[\begin{array}{c c c | c}
                        1 & 0 & 1 & 0\\
                        0 & 1 & 1 & 0\\
                        0 & 0 & 0 & 0\\
                        0 & 0 & 0 & 0
                    \end{array}\right]
            \end{align*}
            Thus $U_1$ is:
            \begin{equation*}
                \text{span} \left(\left[\begin{matrix}
                    -1\\
                    -1\\
                    1
                \end{matrix}\right] \right)
            \end{equation*}
            \begin{align*}
                [A_2 | \vec{0}] &= \left[\begin{array}{c c c | c}
                    3 & -3 & 0 & 0\\
                    1 & 2 & 3 & 0\\
                    7 & -5 & 2 & 0\\
                    3 & -1 & 2 & 0
                    \end{array}\right]\\
                    &\sim \left[\begin{array}{c c c | c}
                        1 & 0 & 1 & 0\\
                        0 & 1 & 1 & 0\\
                        0 & 0 & 0 & 0\\
                        0 & 0 & 0 & 0
                    \end{array}\right]
            \end{align*}
            Thus $U_2$ is also:
            \begin{equation*}
                \text{span} \left(\left[\begin{matrix}
                    -1\\
                    -1\\
                    1
                \end{matrix}\right] \right)
            \end{equation*}
            \begin{enumerate}[label=(\alph*)]
                \item dimensions of $U_1$ and $U_2$ are both 1
                \item bases of $U_1$ and $U_2$ are both shown above
                \item since $U_1$ and $U_2$ have the same basis, the basis of $U_1 \cap U_2$ is:
                    \begin{equation*}
                        \text{span} \left(\left[\begin{matrix}
                            -1\\
                            -1\\
                            1
                        \end{matrix}\right] \right)
                    \end{equation*}
            \end{enumerate}
        \subsection{}
            So for some reason, exercises 2.13 and 2.14 are the same???
        \subsection{}
            \begin{enumerate}[label=(\alph*)]
                \item 
                    \begin{itemize}
                        \item Proving $F$ is a subspace of $\mathbb{R}^3$
                            \begin{itemize}
                                \item When $x, y, z$ are all 0, the vector $\vec{v} \in F$ would be $\vec{0}$. Therefore, $F\neq 0$
                                \item If a vector $\vec{v} = (x, y, z) \in F$, 
                                    then $x + y - z = 0$. 
                                    $\forall \lambda \in \mathbb{R}$, 
                                    $\lambda \vec{v} = (\lambda x, \lambda y, \lambda z)$. 
                                    Since $x + y - z = 0$, 
                                    $\lambda x + \lambda y - \lambda z = \lambda(x + y - z) = \lambda \cdot 0 = 0$. 
                                    Thus, $\lambda \vec{v} \in F$ and $F$ is closed with respect to outer operation
                                \item Let $\vec{v_1} = (x_1, y_1, z_1) \in F$ and $\vec{v_2} = (x_2, y_2, z_2) \in F$. Also, let $\vec{v_3} = \vec{v_1} + \vec{v_2} = (x_1 + x_2, y_1 + y_2, z_1 + z_2) = (x_3, y_3, z_3)$. Since both vectors are in $F$, $x_1 + y_1 - z_1 = 0$ and $x_2 + y_2 - z_2 = 0$. This means that $x_3 + y_3 - z_3 = (x_1 + x_2) + (y_1 + y_2) - (z_1 + z_2) = (x_1 + y_1 - z_1) + (x_2 + y_2 - z_2) = 0 + 0 = 0$. Therefore, $\vec{v_3}\in F$ and $F$ is closed with respect to inner operation
                                \item Since $F$ is not empty and is closed with respect to the inner and outer operations, $F$ is a subspace of $\mathbb{R}^3$
                            \end{itemize}
                        \item Proving $G$ is a subspace of $\mathbb{R}^3$
                            \begin{itemize}
                                \item When $a, b$ are both 0, the vector $\vec{v} \in F$ would be $\vec{0}$. Therefore, $F\neq 0$
                                \item Let $\vec{v} = (a - b, a + b, a - 3b)\in G$. $\forall \lambda \in \mathbb{R}$, $\lambda \vec{v} = (\lambda(a-b), \lambda(a+b), \lambda(a-3b))=(\lambda a - \lambda b, \lambda a + \lambda b, \lambda a - 3\lambda b)$. If we let $a' = \lambda a \in \mathbb{R}$ and $b' = \lambda b \in \mathbb{R}$, then $\lambda \vec{v} = (a' - b', a' + b', a' - 3b')$ which means $\lambda \vec{v} \in G$.
                            \end{itemize}
                    \end{itemize}
                \item When a vector $\vec{v} = (x + y - z)$ is in $F\cap G$, then $x+y-z=0$ and $x=a-b, y=a+b, z=a-3b$ for some $a,b \in \mathbb{R}$. This means that $(a-b) + (a+b) - (a-3b)=a+3b=0$. Therefore, $a=-3b$. This means that $\vec{v} = (-4b, -2b, -6b) = -2b(2, 1, 3)$. Therefore, the basis of $F\cap G$ is:
                    \begin{equation*}
                        \left[\begin{matrix}
                            2\\
                            1\\
                            3
                        \end{matrix}\right]
                    \end{equation*}
                \item 
                    \begin{itemize}
                        \item For $F$, since $x + y - z = 0$, $y$ and $z$ are both free variables so a basis of $F$ is
                            \begin{equation*}
                                \left(
                                    \left[
                                        \begin{matrix}
                                            -1\\
                                            1\\
                                            0
                                        \end{matrix}
                                    \right],
                                    \left[
                                        \begin{matrix}
                                            1\\
                                            0\\
                                            1
                                        \end{matrix}
                                    \right]
                                \right)
                            \end{equation*}
                        \item For $\vec{v} = (x, y, z) \in G$, since $x = (a - b), y = (a + b), z = (a - 3b)$, a basis for $G$ is:
                            \begin{equation*}
                                \left(
                                    \left[
                                        \begin{matrix}
                                            1\\
                                            1\\
                                            1\\
                                        \end{matrix}
                                    \right],
                                    \left[
                                        \begin{matrix}
                                            -1\\
                                            1\\
                                            -3\\
                                        \end{matrix}
                                    \right]
                                \right)
                            \end{equation*}
                        \item If a vector $\vec{v} \in F \cap G$, then there 
                            must be $\vec{x}$ and $\vec{y}$ such that $\vec{v} = F\vec{x} = G\vec{y}$.
                            This means $F\vec{x}-G\vec{y}=\vec{0}$. To solve for $\vec{x}$ and $\vec{y}$, solve the homegeneous equation $Ax=0$ where $A$ is augmented matrix $A = [F, -G]$
                            \begin{align*}
                                A &= \left[
                                    \begin{matrix}
                                        -1 & 1 & -1 & 1\\
                                        1 & 0 & -1 & -1\\
                                        0 & 1 & -1 & 3
                                    \end{matrix}
                                \right]\\
                                &\sim \left[
                                    \begin{matrix}
                                        1 & 0 & 0 & 2\\
                                        0 & 1 & 0 & 6\\
                                        0 & 0 & 1 & 3
                                    \end{matrix}
                                \right]
                            \end{align*}
                            This means that basis of nullspace of $A$ is 
                            \begin{equation*}
                                \left[\begin{matrix}
                                    -2\\
                                    -6\\
                                    -3\\
                                    1
                                \end{matrix}\right]
                            \end{equation*}
                            This means that $\vec{x}$ is any scalar multiple of $(-2, -6)$ and $\vec{y}$ is any scalar multiple of $(-3, 1)$.
                            Therefore, $\vec{v}$ would be any scalar multiple of:
                            \begin{equation*}
                                F\vec{x} = \left[
                                    \begin{matrix}
                                        -1 & 1\\
                                        1 & 0\\
                                        0 & 1
                                    \end{matrix}
                                \right] \cdot \left[
                                    \begin{matrix}
                                        -2\\
                                        -6
                                    \end{matrix}
                                \right] = \left[
                                    \begin{matrix}
                                        -4\\
                                        -2\\
                                        -6
                                    \end{matrix}
                                \right]
                            \end{equation*} 
                            This basis is equal to the one found in (b) since $(-4, -2, -6) = -2(2, 1, 3)$. So there are the same!
                    \end{itemize}
            \end{enumerate}
        \subsection{}
        \subsection{}
        \subsection{}
        \subsection{}
        \subsection{}
            \begin{enumerate}[label=\alph*.]
                \item Done in separate place
                \item Basis change vector $P_1$ can be found by computing the coordinates of the basis vectors of $B$ in terms of $B'$
                    \begin{align*}
                        [B' | b_1] &=
                        \left[\begin{array}{cc|c}
                             2 & 1 & 2\\
                            -2 & 1 & 1
                        \end{array}\right]\\
                        &= \left[\begin{array}{cc|c}
                            1 & 0 & \frac{1}{4}\\
                            0 & 1 & \frac{3}{2}
                       \end{array}\right]\\
                        [B' | b_2] &=
                        \left[\begin{array}{cc|c}
                             2 & 1 & -1\\
                            -2 & 1 & -1
                        \end{array}\right]\\
                        &= \left[\begin{array}{cc|c}
                            1 & 0 & 0\\
                            0 & 1 & -1
                       \end{array}\right]\\
                    \end{align*}
                    Therefore $P_1$ is:
                        \begin{equation*}
                            \left[\begin{matrix}
                                \frac{1}{4} & 0\\
                                \frac{3}{2} & -1
                            \end{matrix}\right]
                        \end{equation*}
                \item 
                    \begin{enumerate}[label=(\roman*)]
                        \item If determinant of $C$ is not equal to 0, then $C$ is a basis of $\mathbb{R}^3$
                            \begin{align*}
                                |C|&=1\left|\begin{matrix}
                                    -1 & 0\\
                                    2 & -1
                                \end{matrix}\right|
                                - 0\left|\begin{matrix}
                                     2 & 0\\
                                    -1 & -1
                                \end{matrix}\right|
                                + 1\left|\begin{matrix}
                                     2 & -1\\
                                    -1 &  2
                                \end{matrix}\right|\\
                                &= 1\times 1 - 0 \times (-2) + 1 \times 3\\
                                &=4 \neq 0
                            \end{align*}
                            Therefore, $C$ is a basis of $\mathbb{R}^3$
                        \item 
                    \end{enumerate}
                \item 
            \end{enumerate}

\end{document}